\documentclass[11pt]{article}
\usepackage[a4paper,margin=1in]{geometry}
\usepackage{graphicx}
\usepackage{booktabs}
\usepackage{amsmath}
\title{Maze Shaping Report: PBRS Manhattan (v2)}
\author{}
\date{\today}
\begin{document}
\maketitle

\section{Experiment Setup}
This report documents the run with the following settings:
\begin{itemize}
\item Potential type: Manhattan distance PBRS
\item Discount factor: $\gamma=0.99$
\item Base step reward: $0.0$
\item Goal reward: $+1.0$ on terminal transition
\item Conditions: \texttt{no\_shaping}, \texttt{phi\_half}, \texttt{phi\_full}
\item Episodes: 500, Runs: 12
\item Validation every 25 episodes, 30 greedy rollouts
\end{itemize}

Reward used for shaping:
\[
R'_{\kappa}(s,a,s') = R(s,a,s') + \gamma\Phi_{\kappa}(s') - \Phi_{\kappa}(s), \quad \kappa\in\{0,0.5,1.0\}.
\]

Here, $\Phi(s)$ is the potential value of state $s$. In this report we use Manhattan potential:
\[
\Phi(s)=-(|r-r_g|+|c-c_g|),
\]
where $(r,c)$ is the agent state and $(r_g,c_g)$ is the goal coordinate.  
Each condition scales this base potential as
\[
\Phi_{\kappa}(s)=\kappa\,\Phi(s),
\]
so $\kappa=0$ is no shaping, $\kappa=0.5$ is half-strength shaping, and $\kappa=1.0$ is full-strength shaping.

\section{Results}
\begin{figure}[h]
\centering
\includegraphics[width=0.85\textwidth]{../../outputs/maze_shaping_icml_style_pbrs_manhattan_v2/learning_curve.png}
\caption{Training curve (steps to goal).}
\end{figure}

\begin{figure}[h]
\centering
\includegraphics[width=0.92\textwidth]{../../outputs/maze_shaping_icml_style_pbrs_manhattan_v2/validation_progress.png}
\caption{Validation progress (success rate and average steps).}
\end{figure}

\section{Final Episode Summary (Episode 500)}
\begin{table}[h]
\centering
\begin{tabular}{lcc}
\toprule
Condition & Validation success rate & Validation mean steps \\
\midrule
\texttt{no\_shaping} & 0.0 & 350.0 \\
\texttt{phi\_half} & 0.0 & 350.0 \\
\texttt{phi\_full} & 0.0 & 350.0 \\
\bottomrule
\end{tabular}
\caption{All conditions remained unsuccessful in this run budget.}
\end{table}

\section{Interpretation}
Under this specific configuration, none of the three conditions reached successful greedy validation behavior within 500 episodes. The run is still useful as a record of a negative result for this reward design.

\section{Artifacts}
\begin{itemize}
\item Output folder: \texttt{../../outputs/maze\_shaping\_icml\_style\_pbrs\_manhattan\_v2}
\item Learning CSV: \texttt{learning\_curve.csv}
\item Validation CSV: \texttt{validation\_progress.csv}
\item Run summary: \texttt{run\_summary.json}
\item GIFs: \texttt{gifs/policy\_rollout\_ep\_*.gif}
\end{itemize}

\end{document}
